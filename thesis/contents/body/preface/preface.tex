\newpage
\chapter*{\Large MỞ ĐẦU}
\addcontentsline{toc}{chapter}{MỞ ĐẦU}
% \addtocontents{toc}{\protect\contentsline{section}{Lời mở đầu}{}{}}
\section*{Tính cấp thiết của đề tài}
\addcontentsline{toc}{section}{Tính cấp thiết của đề tài}
Ngày nay, nhiệm vụ phân loại trong học máy (machine learning) đóng một vai trò quan trọng trong nhiều ứng dụng thực tế, giúp giải quyết các vấn đề thực tiễn và cải thiện hiệu quả của nhiều quy trình. Ví dụ:
\begin{itemize}
    \item \textbf{Ứng dụng trong y tế}: Trong y học, phân loại giúp xác định các bệnh, chẳng hạn như phân loại hình ảnh chẩn đoán (MRI, X-quang) để phát hiện các bệnh ung thư, bệnh tim mạch, hoặc các vấn đề sức khỏe khác. Sự chính xác trong phân loại có thể cứu sống bệnh nhân.
    \item \textbf{An ninh và bảo mật}: Trong nhận dạng khuôn mặt, nhận dạng vân tay, hay các hệ thống xác thực sinh trắc học, phân loại giúp xác định người dùng một cách nhanh chóng và chính xác. Điều này rất quan trọng trong bảo mật thông tin cá nhân và giao dịch.

    \item \textbf{Phân loại thư rác (Spam)}: Trong các ứng dụng email, phân loại thư điện tử thành thư rác và thư hợp lệ là một trong những nhiệm vụ phân loại quan trọng, giúp người dùng tránh bị quấy rầy bởi các thông tin không mong muốn.

    \item \textbf{Xử lý ngôn ngữ tự nhiên (NLP)}: Các nhiệm vụ phân loại trong NLP như phân loại cảm xúc trong văn bản, phân loại chủ đề hay phân loại văn bản theo các chủ đề cụ thể là rất quan trọng trong các ứng dụng như phân tích cảm xúc, chatbot, và các hệ thống tìm kiếm.

    \item \textbf{Tiếp thị và phân tích khách hàng}: Phân loại khách hàng theo các nhóm dựa trên hành vi mua sắm, sở thích hoặc mức độ gắn kết giúp các doanh nghiệp tối ưu hóa các chiến lược marketing và gia tăng hiệu quả tiếp cận khách hàng.

    \item \textbf{Tự động hóa và phân tích dữ liệu lớn}: Phân loại dữ liệu là một trong những nhiệm vụ cơ bản trong phân tích dữ liệu lớn, giúp nhận diện các mẫu dữ liệu quan trọng và phân loại chúng thành các nhóm có ý nghĩa, từ đó hỗ trợ ra quyết định.

    \item \textbf{Quản lý rủi ro trong tài chính}: Phân loại các khoản vay là rủi ro thấp hay cao giúp các tổ chức tài chính đưa ra các quyết định cho vay, giảm thiểu rủi ro tài chính.
\end{itemize}
Các bộ phân loại tuyến tính từng phần là một mở rộng đơn giản của các bộ phân loại tuyến tính, đồng thời có khả năng xấp xỉ các ranh giới phi tuyến tính. Điều này làm cho chúng trở nên phù hợp cho các ứng dụng đòi hỏi tốc độ xử lý nhanh và ít tốn bộ nhớ, chẳng hạn như robot trinh sát nhỏ, camera thông minh, và các hệ thống nhúng.

\section*{Mục đích và nhiệm vụ nghiên cứu}
\addcontentsline{toc}{section}{Mục đích và nhiệm vụ nghiên cứu}
\subsection*{Mục đích nghiên cứu}
\addcontentsline{toc}{subsection}{Mục đích nghiên cứu}

\subsection*{Nhiệm vụ nghiên cứu}
\addcontentsline{toc}{subsection}{Nhiệm vụ nghiên cứu}
Nhiệm vụ nghiên cứu của đề án bao gồm:

\section*{Đối tượng và phạm vi nghiên cứu}
\addcontentsline{toc}{section}{Đối tượng và phạm vi nghiên cứu}
\textbf{Đối tượng nghiên cứu}

\textbf{Phạm vi nghiên cứu}g

\section*{Cơ sở lí luận và phương pháp nghiên cứu}
\addcontentsline{toc}{section}{Cơ sở lí luận và phương pháp nghiên cứu}
\textbf{Cơ sở lý luận}

\textbf{Phương pháp nghiên cứu} được áp dụng bao gồm:

\section*{Kết cấu của đề án}
\addcontentsline{toc}{section}{Kết cấu của đề án}
Ngoài phần mở đầu, kết luận, tài liệu tham khảo, phụ lục đề án có 3 chương.
\begin{itemize}
    \item[-] Chương 1: KIẾN THỨC CHUẨN BỊ
    \item[-] Chương 2: THUẬT TOÁN SVM VỚI ÁNH XẠ ĐẶC TRƯNG TUYẾN TÍNH TỪNG PHẦN
    \item[-] Chương 3: ỨNG DỤNG CỦA THUẬT TOÁN SVM VỚI ÁNH XẠ ĐẶC TRƯNG TUYẾN TÍNH TỪNG PHẦN
\end{itemize}